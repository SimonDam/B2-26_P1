K–12 is a term for the sum of primary and secondary education. 

The Maine Education Policy Research Institute (MEPRI), the Center for Education Policy, Applied Research and Evaluation (CEPARE) at the University of Southern Maine, og the Nellie Mae Education Foundation (NMEF), har i samarbejde med hinanden prøvet at forbedre K-12 skolesystemet i USA, da staten ønsker at eleverne har større chance for at udleve deres drømmeuddannelse, hvilket ikke var tilfældet i daværende tidspunkt. Kort tid efter organisationerne har retledt K-12 skolesystemet, er der klare forbedringer, i form af et gennemsnit som er højere end landsgennemsnittet samt bedre udbytte generelt. 

Disse resultater udgøres bl.a. af elevernes forøgede ydelse, som skolerne har formået at frembringe gennem korrekt brug af skolernes ressourcer. Pånævnte skoler har eksperimenteret med faktorer, som i sidste ende har haft en positiv virkning på elevernes resultater. Herunder er en måden, hvorpå undervisningen udelukkende fokuserer på elevernes udbytte en essentiel faktor. Ligeledes prøver skolen at få input fra eleverne via. spørgeskemaer eller lignende, for at udpege problemstillinger, som måske hindrer enkelte elevers præstationsevne hvad end det er socialt, kulturelt mm. 

Dog er dette forskelligt fra skole til skole, og derfor er det nødvendigt for skoler, som har brug for forbedringer at foretage selvevaluveringer for at opnå ligeledes resultater. Skolerne kan henholdsvis benytte spørgeskemaer eller lignende, men i sidste ende er det en proces som er anderledes fra skole til skole, og derfor er det vigtigt, at undersøge hver skole på hver sin måde, og ikke blot tage stilling til en anden skoles opskrift. 

