Programmet bruger structs, så strukturen af dataene, der produceres, er lettere at håndtere. Tre structs, ”individual”, ”teacher” og ”requirements” er brugt til henholdsvis skemaerne, information om lærerne og -kravene til en klasse.

\begin{lstlisting}[language = C]
struct individual{
  int lesson_num[LESSONS_PER_DAY_MAX][SCHOOL_DAYS_IN_WEEK];
  int fitness;
  int perfection;
  int lessons_with_parallel;
  int lessons_with_both;
  int heavy_lesson_after;
  int heavy_lesson_before;
  int teacher_overbooked;
  int best_gena7;
  int best_gena8;
  int best_gena9; 
};
\end{lstlisting}

”Individual” består af et array af arrays af integers, lesson\_num, der repræsenterer lektionnummeret (faget) for hver blok på hver dag. Fitness indeholder skemaets fitnessværdi. Perfection indeholder antallet af fag, der optræder nok gange i skemaet ud fra kravene, og som ikke har mere end en blok for meget. lessons\_with\_parallel indeholder antallet af gange, skemaet har samme fag som en parallelklasse. lessons\_with\_both angiver antallet af gange, skemaet har samme fag som begge parallelklasser. heavy\_lesson variabler angiver, hvor mange gange de tunge fag ligger før og efter middag. teacher\_overbooked fortæller hvor mange gange, en lærer i skemaet bliver brugt i en anden parallelklasse på samme tid. best\_gen variablerne angiver i hvilken generation, de bedste skemaer for hver årgang er opstået.

\begin{lstlisting}[language = C]
struct class_info{
  int number_of_lessons;
  char padding[2];
  char teacher_name[TEACHER_NAME_MAX];
  char lesson_name[LESSON_NAME_MAX];
  char class_name[TEACHER_NAME_MAX];  
};
\end{lstlisting}

“class_info” structen indeholder navnet på læren, navnet på faget, antallet af lektioner læren har for det givne fag og klassen læren har til faget. Denne struct skal forstås sådan, at en klasse med et specifikt fag har et specifikt antal undervisningstimer med en specifik lære. Som eksempel, 7.a kunne have 4 timers matematik med læren ”JP”. 

\begin{lstlisting}[language = C]
struct requirements{
  int Dan_req;
  int Mat_req;
  int Eng_req;
  int Tys_req;
  int Fys_req;
  int His_req;
  int Sam_req;
  int Val_req;
  int Geo_req;
  int Bio_req;
  int Gym_req;
  int Fri_req;
  int Rel_req;
  int Pra_req;
};
\end{lstlisting}

”requirements” er en struct udelukkende bestående af heltal. Disse heltal viser kravene til de forskellige fag. Det er disse krav, som programmet blandt andet bruger til at tjekke om skemaet er opfyldt, eller om der mangler et specifikt fag, som skal fyldes ind i skemaet.
\newpage