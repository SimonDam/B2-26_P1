Programmet bruger structs, så strukturen af dataene, der produceres, er lettere at håndtere. Fire structs, ”lesson”, ”individual”, ”teacher” og ”requirements” er brugt til henholdsvis en lektionens information, skemaets information, lærerens information og og de faglige krav.

\begin{lstlisting}[language = C]
struct lesson{
  char teacher_name[TEACHER_NAME_MAX];
  char lesson_name[LESSON_NAME_MAX];
};
\end{lstlisting}

Structen, ”lesson”, består af two variabler, et navn på læreren og et navn på lektionen. Hovedsageligt bliver denne funktion brugt til at printe skemaet ud.

\begin{lstlisting}[language = C]
struct individual{
  int individual_num[LESSONS_PER_DAY_MAX][SCHOOL_DAYS_IN_WEEK];
  int fitness;
  int grade;
};
\end{lstlisting}

”Individual” består af integer array af arrays, individual\_num, og to integers, fitness og grade. Array’et har to indgange, hvor det første identificerer, hvilken lektion skal tilgås og den anden tilgår ugedagen. Hver indgang i skemaet er givet ved et heltal, som korrelerer med et fag. Fitness er en værdi, der udregnes ud fra, hvor godt skemaet opfylder de krav, der stilles af skemalæggeren. Det er denne værdi, som bestemmer skemaets bæredygtighed i forhold til andre skemaer. Grade er et heltal, som bestemmer hvilken klasse, der er tale om. Siden forskellige krav stilles til forskellige klassetrin, er det vigtigt for programmet, at være i stand til at skelne mellem klassentrinnene.

\begin{lstlisting}[language = C]
struct teacher{
  char teacher_name[TEACHER_NAME_MAX];
  char lesson_name[LESSON_NAME_MAX];
  int number_of_lessons;
  char class_name[TEACHER_NAME_MAX];
};
\end{lstlisting}

“teacher” structen indeholder navnet på læreren, navnet på faget, antallet af lektioner læreren har for det givne fag og klassen læreren har til faget. Denne struct skal forstås sådan, at en klasse med et specifikt fag har et specifikt antal undervisningstimer med en specifik lære. Som eksempel, 7.a kunne have 4 timers matematik med læreren ”Jørgen”. 

\begin{lstlisting}[language = C]
struct requirements{
  int Dan_req;
  int Mat_req;
  int Eng_req;
  int Tys_req;
  int Fys_req;
  int His_req;
  int Sam_req;
  int Val_req;
  int Geo_req;
  int Bio_req;
  int Gym_req;
  int Fri_req;
  int Rel_req;
  int Pra_req;
};
\end{lstlisting}

”requirements” er en struct udelukkende bestående af heltal. Disse heltal afspejler minimumsantallet lektioner for de forskellige fag. Det er disse krav, som programmet bl.a. bruger til at tjekke om skemaet er opfyldt, eller om der mangler et specifikt fag, som skal fyldes ind i skemaet.

Hvordan disse krav bruges vil bruges i kodeeksempler senere i afsnittet.
