På grund af tidspres, eller manglende evner, var der nogle ting der ikke blev implementeret i løsningen. I det følgende afsnit, vil disse mangler blive pointeret. 
\subsubsection{Mangler i løsningen}

I løsningen er der ingen lærer der har lektioner på mere end et klassetrin. Dette er en mangel, da lærerne på en folkeskole arbejder på tværs af klassetrinene. Dette blev ikke implementeret, da alle klassetrinene skal læses ind på samme tid, for at undersøge om lærerne er sat til at være mere end et sted ad gangen. I løsningen indlæses klassetrinene hver for sig, men parallel klasserne indlæses på samme tid. Det undersøges altså stadig om lærerne er mere end et sted ad gangen, dog ikke på tværs af klassetrinene. 
Derudover er der kun blevet arbejdet med udskoling. Der mangler derfor at blive generet skemaer, for klasserne i indskoling mellemskoling. For at kunne genere skemaer for indskoling og mellemskoling, skal der indsættes nye ’requirements’ for disse klassetrin.
\subsubsection{Graphic user interface}

Hvis programmet reelt skulle være et produkt der kunne sælges, ville det kræve en gui, eller graphic user interface. Hvis brugeren skal kunne ændre den fil der bliver læst ind, og de ændringer de laver skal være formateret på samme vis, som det der er i dokumentet nu. Hvis brugere ændrer dokumentet og det ikke står i samme format, vil filen ikke blive læst ind korrekt og programmet vil lukke. En reel gui ville derfor være den bedste måde hvorpå brugeren kan udføre input der af programmet nedskrives i en fil. For at opstille den bedst mulige gui, ville man skulle undersøge brugervenlighed, for at finde det bedst mulige design. I gui’en skulle brugeren have mulighed for at ændre antallet af lærere, deres navn/initialer samt hvor højt de ville prioritere forskellige bindinger. 
Det aktuelle program er programmeret til Sofiendahlskolens præferencer, og ville derfor ikke kunne bruges på en skole, hvor de vægter f.eks. senere møde timer højt. 


Derudover er der heller ikke taget højde for lokale reservering i det aktuelle program. Brugeren vil derfor være nødsaget til selv at uddele lokaler efter genereringen af skemaet. I videreudviklingen af programmet, ville en forbedring kunne være at hver klasse bliver tildelt et lokale samt en lærer, og at lokalerne i nogle tilfælde skulle tilhører de forskellige klasser, eller fag, således at idræt f.eks. kun kan foregå i idrætshallen. På den måde, ville det også kunne bestemmes hvilke klasser der for eksempel kan have idræt sammen. Derudover er antallet af parralelklasser, antallet af fag og antallet af klassetrin ikke fleksibelt, så det kan bruges på andet end en udskoling med tre parrelelklasser. Programmet kan delvist tilpasses til nye krav ved ændring af defines i source koden. Ved videreudvikling vil denne del gøres me fleksibel.


En log ind mulighed for skolerne ville også kunne forbedre programmet. At brugeren kan logge ind, kan muliggøre at skemaerne kan gemmes på en profil. Det vil derfor ikke være nødvendigt generer et nyt skema, hver gang programmet køreres. I det aktuelle program er brugeren nød til at gemme en kopi af det generede skema på computeren manuelt for at kunne gemme det. En log ind mulighed vil endvidere kunne gemme skolens præferencer, således at præferencerne ikke skal skrives ind hver gang der skal genereres et nyt skema. 
\subsubsection{Brugertest}

Hvis en gui bliver implementeret i programmet, ville det også have været relevant at foretage brugertests. En brugertest ville forsikre at gui’s interface er brugervenligt, og overskueligt. I interviewet der blev foretaget med Sofiendahlskolen, blev det pointeret at de førhen havde forsøgt sig med softwareløsninger til planlægningen af deres skema. De programmer de havde brugt syntes de dog var for besværlige at sætte sig ind i, de endte derfor med ikke at fortsætte med softwareløsningerne.\footfullcite{interview} Det er derfor vigtigt at programmet er brugervenligt, da skolerne ellers ikke vil fortsætte med at bruge produktet. 
\subsubsection{Empiri fra flere skoler}

For at kunne programmere en bedre og mere fleksibel softwareløsning til skemalægningsprocessen, skal der samles mere empiri fra flere skoler. Den aktuelle løsning er begrænset, i det den kun løser Sofiendahlskolens problemer, og kun tager højde for deres bindinger. Hvilke bindinger de forskellige skoler, har og hvordan de prioriterer dem er meget forskelligt. Da løsningen kun tager højde for Sofiendahlskolens bindinger, kan løsningen ikke bruges optimalt på andre folkeskoler, medmindre, de har præcis de samme bindinger som Sofiendahlskolen. At indsamle mere empiri ville også hjælpe med at danne et større perspektiv over de problemer, der forekommer i skemalægningsprocessen på andre skoler. 

For at skabe mere brugervenlighed og effektiv skemalægning for skolernes personale, kunne programmet også udfra en valgt kommune af brugeren, sørger for hvert år automatisk at opdatere kravene til timeantal osv. udfra kommunernes og undervisningsministeriets krav.

I gui'en kunne det også designes så, brugeren kunne flytte lektioner, ændre lærere og lokaler på lektioner osv. hvorefter programmet ville vise, hvilke konflikter ændringen ville skabe og/eller planlægge mulige ændringer, så kravene stadig opfyldes.
