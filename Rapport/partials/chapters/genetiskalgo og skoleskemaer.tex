\subsection{Genetiske algoritmer of skoleskemaer}
Et skoleskema består af grundlæggende elementer. Som eksempel kan lektioner være et sådanne element, mens tidspunktet for lektionen kan være et andet. Disse grundlæggende elementer er forholdsvist nemme at beskrive, da de er konkrete, og som sagt før, gundlæggende for skoleskemaets opbygning. Der er også mere abstrakte elemtenter i et skoleskema, som hvornår hvilke fag skal lægge tidsmæssigt, både i forhold til hinanden og i forhold til tiden. Det er disse elementer, som kræver mere individuel input, hvis de skal opfylde de krav, som den enkelte bruger skal bruge.

Genetiske algoritmer giver derfor mulighed for let at generere et skoleskema. Som nævnt tidligere bliver mange generationer genereret ud fra parametrer og deres vigtighed overfor 'fitness'-værdien. Dette betyder, at specifikke parametrer kan tages højde for, og det giver større mulighed for fleksibilitet uden at kræve en omvæltning at selve programmet. 

Det er dette koncept, som bærer grundlaget for brugen af genetiske algoritmer til genereringen af skoleskeamer, da programmet helt naturligt speciferer sit resultat til at tilfredsstille de krav, som bliver stillet af brugeren. Dette, samlet med princippet om at kravenes vigtig kan justeres nemt, giver stærkt grundlag for brugen af genetiske algoritmer.

