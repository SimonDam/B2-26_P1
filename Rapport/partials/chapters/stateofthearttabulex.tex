Tabulex er et dansk skemalægningsprogram som fungerer ved at brugeren først indtaster alle sine ressourcer (lærere, lokaler, klasser, fag og placeringsregler). 
Under hver enkelt lærer sættes maksimum antal mellemtimer, altså timer hvor de ikke underviser. Derudover vælges maksimum og minimum antal lektioner per dag på specifikke ugedage samt blokeringer der gør at læreren ikke kan have undervisninger på bestemte tidspunkter. 
Samme parametre kan sættes for klasserne, men også tre nye. Et parameter angiver hvor tidligt eleverne som krav skal møde hver dag. De kan altså godt møde tidligere nogle dage, men ikke senere. En anden parameter angiver efter samme princip, hvornår eleverne tidligst får fri. Sidste parameter angiver om, det er et krav, at klassen skal møde hver dag. Hvis lærere har arbejdsopgaver udenfor klasserne, er det nødvendigt at lave fiktive klasser uden elever dertil.
Under hold er det muligt at lave et hold af flere klasser.
Under fag, indtastes fagene til klasserne enkeltvis. Der kan sættes et krav til, at en klasse ikke må have fagrepetition, altså to blokke af samme fag på en dag. Som udgangspunkt forsøger Tabulex at undgå fagrepetition, men hvis dette krav ikke er valgt, tillades det i nogen situationer. Fagene kan også ligges i faggrupper, hvor Tabulex også vil forsøge at undgå repetition af fag i samme faggruppe.
Lokaler bliver inddelt i grupper, efter hvilke fag de kan bruges til. Derudover har de også en blokeringsmulighed, hvis lokalerne ikke kan bruges på specifikke tidspunkter.
Der findes fire placeringsregler i Tabulex, men flere kan oprettes manuelt. De eksisterende placeringsregler kan bruges til at bestemme, at binding skal ligge i første lektion, at den skal ligge i sidste lektion eller at den skal ligge i enten første eller sidste lektion. Den sidste placeringsregel bestemmer at en lektion ikke må være skemalagt to dage i træk.
Når alle ressourcer er defineret, sættes de sammen med bindinger med lærer, klasse, fag, lokale og eventuelt placeringsregel. Når brugeren laver bindingerne, kan der løbende følges med i, hvor mange lektioner der er afsat til hver lærer, klasse og lokale.
Når disse informationer er indtastet manuelt, kan den automatiske skemalægningsfunktion bruges. Denne funktion leder efter fejl i det manuelle skema, og forsøger derudfra at oprette bedre skema. Under denne proces kigges på parametre som antallet af mellemtimer og overholdelse af krav om undervisning hver dag, mødetider, gåtider, antal lektioner på en dag og antallet af blokeringer overtrådt.\footfullcite{tabulex}
