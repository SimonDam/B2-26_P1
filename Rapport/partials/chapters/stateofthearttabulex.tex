Skemalægningsprogrammerne er dog ikke problemfri. Selvom programmet opfylder de mest væsentlige krav omkring, hvorvidt et skema bør lægges for at få optimalt udbytte, er programmet for upræcist i forhold til hvilke parametre der tages stilling til, og hvilken af parametre prioriteres højest. Typisk vil sådan et program virke for en skole, hvor lærere ikke har problemer med hensyn til opdeling i teams mm., men dette er ikke tilfældet nogle steder. Heriblandt er Sofiendalskolen, som er af skolerne, hvor lærernes teams ikke fungerer optimalt på grund af, nogle af lærerne er medlemmer i flere teams. Dette er en essentiel parameter som der ikke tages højde for i skemalægningsprogrammerne, som forårsager forringet udbytte af programmet og i værste fald en helt anden alternativ metode at lægge skemaet. Dog er der samtidig andre faktorer, som gør det en hel del svære at benytte skemalægningsprogrammerne, da programmerne ellers skal være skræddersyet for en specifik skole, før det kan lade sig gøre.
Derfor vil en mulig forbedring af de nuværende skemalægningsprogrammer være at tage stilling til mindre faktorer. Programmet skal derfor ikke kunne udlevere et endeligt skema, men til gengæld skal det kunne give en klar formular eller en retningslinje, som skolen herefter kan følge og tilpasse, alt afhængigt af hvilke parametre og faktorer skolen prioriterer højest\cite{tabulex}. 
