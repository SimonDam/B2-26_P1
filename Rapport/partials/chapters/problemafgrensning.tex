I interviewet med Søren Kusk fra Sofiendalskolen, er information indsamlet, hvilke krav skolen stiller, når skemaet lægges, samt hvilke problemer der opstår under skolens skemalægning. Derudover er det blevet undersøgt, hvorfor der bruges manuel skemalægning fremfor at bruge en af et allerede eksisterende softwareløsninger på markedet.\footfullcite{interview}
\\\\
Ud fra empiri fra interviewet samt research af state of art, kan det konkluderes, at de eksisterende skemaplanlægningsprogrammer på markedet har svært ved at tage hensyn til Sofiendalskolens krav.
\\\\
For Sofiendalskolens lærere er det nødvendigt at indgå kompromisser, og det er svært at standardiserer vigtigheden af de enkelte parametre, der skal tages højde for i skemaplanlægningsprocessen. Derudover er det en tidskrævende proces for lærerne, at sætte sig ind og få forståelse for et af de allerede eksisterende skemaplanlægningsprogrammer.
\\\\
Ud fra denne viden er det blevet besluttet at lave et program, der kan lave et grundskema, som stemmer overens med de lovgivningsmæssige krav for elevernes timetal. Skemaet skal tage højde for, at hver klasse og lærere ikke kan have mere end en lektion af gangen. Ved brug af generisk algoritmer laves et fitness niveau, der tilpasses af i hvor høj grad det lykkedes at lægge lærernes forberedelsestimer i forlængelse af hinanden, så de har 2-3 timers forberedelse af gangen, da dette vægter højt for Sofiendalskolen. Derudover skal programmet være i stand til at vurderer fitness på tværs af klasser, så programmet giver mulighed for at vægte skeamer med tværfaglige lektioner højere.\footfullcite{interview}

Programmet vil løse en del af det tidsmæssige problem for lærerne samt problemet med fordelte forberedelsestimer.