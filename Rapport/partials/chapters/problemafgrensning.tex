I interview med Søren Kusk fra Sofiendalskolen, har vi fået information om, hvilke krav de stiller, når der ligges skema på Sofiendalskolen. Samt hvilke problemer der opstår under skolens skemalægning. Derudover er det blevet undersøgt hvorfor der bruges manuel skemalægning fremfor at bruge en af et allerede eksisterende softwareløsninger på markedet.\footfullcite{interview}
Ud fra vores empiri fra interviewet samt egen research af state of art, har vi fundet ud af, at de eksisterende skemaplanlægningsprogrammerne på markedet har svært ved at tage hensyn til folkeskoleskema, sådan lærere har tidsmæssigt samlede forberedelsestimer.

På sofiendalskolen føler de at det vil være nødvendigt, at indgå kompromiser, og det er svært at standardiserer vigtigheden af de enkelte parametre der skal tages højde for i skemaplanlægningsprocessen. Derudover er det en tidskrævende proces for lærerne, at sætte sig ind og få forståelse for et af de allerede eksisterende skemaplanlægningsprogram.
Ud fra denne viden, har vi besluttet at lave et program der kan lave et grundskema, der stemmer overens med de lovgivningsmæssige krav for elevernes timetal. Vores skema skal tage højde for at hver klasse og lærer ikke kan have mere end en lektion af gangen, samt et begrænset antal faglokaler. Vi vil ved brug af generisk algoritme lave et fit niveau der tilpasses af, i hvor høj grad det lykkedes at lægge lærernes forberedelsestimer i forlængelse af hinanden, så de har 2-3 timers forberedelse af gangen, fremfor mange små forberedelsestimer, da dette vægter højt for Sofiendalskolen.\footfullcite{interview}
Så vores program vil løse en del af det tidsmæssige problem for lærerne samt problemet med fordelte forberedelsestimer.