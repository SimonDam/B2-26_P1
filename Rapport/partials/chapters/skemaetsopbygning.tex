I Danmark følger folkeskolerne et fastlagt skema, der giver struktur til eleverne og lærernes dagligdag. Skemaerne for hver folkeskole er unikke, da de er selvstændigt udarbejdet, det er derfor forskelligt hvilke parametre skolerne prioriter. Prioriteterne kan ændres individuelt mellem skolerne, f.eks. kan nogle skoler prioritere at have flere idræts timer, mens andre prioriterer at have fagene i en hvis rækkefølge. Alle folkeskoleskemaer skal opfylde krav, som er opstillet af regeringen, f.eks. er der et minimumskrav for hvor mange lektioner eleverne på hvert klassetrin skal have i et helt skoleår, se afsnittet ”Lovmæssige krav til folkeskoleskemaet”. Kommunerne skolerne ligger i influerer også på skemaplanlægningen. F.eks. har folkeskolerne i Aalborg kommune flere lektioner end folkeskolerne i Rebild kommune. Skoleskemaet er bygget op således, at eleverne har en række fag hver dag med pauser ind imellem fagene. Lektionerne varer typisk 45 minutter, bemærk at det ikke er det samme som undervisningstimer defineret af uddannelsesministeriet, med pauser på 15 minutter ind imellem lektionerne og en lang middagspause. Nogle skoler vælger dog at afvige fra denne formular ved, f.eks. at have lektioner på 90 minutter med længere pauser ind i mellem. Derudover har lærerne forberedelsestimer, når de ikke underviser. Det vil sige at en lærers forberedelsestime potentielt kunne lægge mellem to lektioner, de skal undervise i. Da der er undervisningspligt i 10 år i Danmark, kan folkeskolen være nødsaget til at planlægge 30 skoleskemaer hvert skoleår\cite{lov2016}. Da der er mange parametre og krav den skemaansvarlige skal tage stilling til, kan skemaplanlægningsprocessen være langvarig\cite{interview}.