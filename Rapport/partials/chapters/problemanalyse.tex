I følgende afsnit redegøres for de metoder og valg, der er taget i forhold til problemanalysen. En teknologianalyse i ”State of the art”-afsnittet tager udgangspunkt i allerede eksisterende programmer, samt laves der en vurdering af potentielle mangler eller problemer med disse programmer.
\\\\
De konkrete krav, som stilles fra uddannelsesministeriet, er forklaret i et lovgivningsafsnit. Afsnittet vil fremstille de krav, der er nødvendig at opfylde, når løsningen skal laves. Kravene beskrevet i dette afsnit er nødvendige for forståelsen af opbygningen af folkeskoleskemaer. I afsnittet tages der udgangspunkt i de kriterier, som beskrives i ”Bekendtgørelse af lov om folkeskolen” med supplerende kilder.
\\\\
Interessentanalysen vurderer de påvirkende og påvirkede interresenter i forhold til skemalægningen. Afsnittet lægger fokus på vurderingen af interessenternes rolle i skemalægningsprocessen. Analysen udarbejdet i dette afsnit giver grundlag for valget af kontaktede interresenter.
\\\\
En vital del af problemanalysen er case-study’en, som blev lavet med Søren Kusk fra Sofiendalskolen. Interviewet blev semikonstrueret med forberedte spørgsmål, hvorefter der blev stillet uddybende spørgsmål løbende i interviewet. Det var væsentligt at snakke med en, som havde indflydelse på skemalægningen samt forstod og kunne formidle de problemer, som kunne opstå ved denne proces.
\\\\
Problemanalysens mål er, at forstå skemaets konstruktion samt at finde de mulige problemer, der kan opstå ved selve processen. Underafsnittende præsenteret i dette afsnit vil analyseres og udvides med tilstrækkelig information, så en fyldestgørende problemafgrænsning kan konstrueres. 
