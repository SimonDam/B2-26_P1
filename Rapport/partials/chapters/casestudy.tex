For at få et aktuelt indblik i nogle af de problemstillinger, der opstår når en skole lægger et skema, blev der foretaget et interview med en af skemalæggerne fra Sofiendalskolen.
\\\\
Sofiendalskolen blev opført i 1911. I dag er Sofiendalskolen en tresporet skole med en special ADHD klasse, hvilket betyder at der på en årgang befinder sig 3 klasser a, b og c samt speciel klassen. Skolen rummer 70 lærere og pædagoger, som underviser 650 elever dagligt.\footfullcite{interview}
\\\\
Lørdag den 27. oktober blev Søren Kusk interviewede. Søren Kusk fungerer som matematiklærer samt it-ansvarlig på Sofiendalskolen og indgår i et team af 8 lærer, som underviser 3. klasse. To af disse lærere underviser også i andre klasser. På Sofiendalskolen er det ikke en bestemt person, der er ansvarlig for at lægge skemaet. Derimod mødes alle involverede pædagoger og lærere to onsdage før starten af skoleåret. Her afholdes to møder med en varighed på tre timer hver, hvilket vil sige, at det tager ca. 420 mandetimer for Sofiendalskolen at lægge skoleårets skema. Skemalægningsprocessen er et meget simpel koncept, men en kompliceret process på Sofiendalskolen. Lærerne og pædagogerne lægger skema uden brug af computerprogrammer. Lærerne starter med at få tildelt hvilke fag, klasser og antal lektioner, de skal undervise. Herefter får de hver især farvede brikker, som repræsenterer de lektioner som lærerne underviser. Brikkerne lægger de på et tomt skema ud fra de klasser, de skal undervise i, indtil de ikke har flere brikker.\footfullcite{interview}
\\\\
Når skemaet lægges på Sofiendalskolen ønskes det, at lærerne har sammenhængende forberedelses timer, således de ikke er spredt ud over hele ugen. De mener, at de ikke får chancen for at forberede sig ordentligt, hvis de kun har en forberedelsestime af gangen. Derudover prioriterer de, at eleverne ikke har tunge fag, som f.eks. matematik over middag, da eleverne tit er trætte sidst på dagen, og derfor får begrænset udbytte af undervisningen. Derudover vil lærerne gerne have mulighed for at lave tværfaglig undervisning på tværs af klasserne, hvilket vil sige, at alle tre parallelklasser a, b og c f.eks. skal have mulighed for at have dansk på samme tidspunkt hver mandag. For skemaplanlæggerne på Sofiendalskolen er det besværligt at opfylde alle disse tre kriterier på en gang. Derfor går de ofte på kompromis med parametrene, og vælger hvilke parametre de prioriterer højst. Skemaplanlæggerne bruger på nuværende tidspunkt ikke skemalægningsprogrammer, da programmerne, de har afprøvet, ikke har kunnet tage højde for flere parametre og præferencer, spredt ud over de forskellige teams og klasser.\footfullcite{interview}
\\\\
Sofiendalskolen har prøvet at lægge skema med programmet Tabulex. Skemaplanlæggerne synes Tabulex var komplekst og avanceret at opstille for hver klasse, da hvert team har forskellige præferencer. Det var derfor svært for Tabulex at lægge skemaet, da Tabulex ikke var fleksibelt nok til deres behov. Derfor har Sofiendalskolen valgt at planlægge skemaerne manuelt. Dog vil Sofiendalskolen stadig være åben for brugen af et computerprogram til skemalægningsprocessen, hvis programmet kunne hjælpe dem med at gøre processen kortere uden at gøre det for avanceret at få sat op.\footfullcite{interview}

