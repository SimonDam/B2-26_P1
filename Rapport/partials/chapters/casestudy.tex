For at få et aktuelt indblik i nogle af de problemstillinger der opstår når en skole lægger et skema, har vi fremstillet et interview med en af skemalæggerne fra Sofiendalskolen.

Sofiendal skolen blev opført i 1911. I dag er Sofiendal skolen en tresporet skole med en speciel adhd klasse hvilket betyder at der på en årgang befinder sig 3 klasse a, b og c samt speciel klassen. Skolen rummer 70 lærer samt pædagoger, som underviser 650 elever dagligt.\footfullcite{interview}

Lørdag den 27 oktober interviewede vi Søren Kusk. Søren Kusk fungerer som matematiklærer samt it-ansvarlig på sofiendalskolen og indgår i et team af 8 lærer som underviser 3. klasse, hvor 2 af dem også underviser en anden klasse. På sofiendalskolen er det ikke en bestemt person, som er ansvarlig for at lægge skemaet, men derimod mødes alle involverede pædagoger og lærer 2 onsdage i starten i året. Her afholdes 2 møder med en varighed på 3 timer, hvilket vil sige det tager ca 420 mandetimer for sofiendalskolen at lægge årets skema. Skemalægningen er en meget simpel, men kompliceret proces på sofiendalskolen. Lærerne og pædagogerne lægger skema uden brug af computerprogrammer. Lærerne starter med at få tildelt hvilke fag, klasser og hvor mange timer de skal have. Herefter får de hver især farvede brikker, som repræsenterer de timer som lærerne har. brikkerne ligger de på et tomt skema ud fra de klasse, som de skal underviser indtil de ikke har flere brikker.\footfullcite{interview}

Når der lægges skema på sofiendalskolen ønskes det, at lærerne har sammenhængende forberedelses timer, således de ikke er spredt ud over hele ugen, da de mener de ikke får chancen for at forberede sig ordentligt, hvis de kun har en forberedelses time af gangen. Derudover prioriterer de at eleverne ikke har tunge fag, som f.eks matematik, over middag, da eleverne tit er trætte på daværende tidspunkt og derfor får begrænset udbytte af undervisning. Derudover vil lærerne gerne have mulighed for at lave tværfaglig undervisning på tværs af klasserne, hvilket vil sige at alle 3 parallelklasser a, b og c f skal have mulighed for at have dansk på samme tidspunkt hver mandag. Skemaplanlæggerne på Sofiendalskolen føler at det er besværligt at opfylde alle disse 3 kriterier på en gang, derfor går de på kompromis med parametrene og vælger hvilke de prioriterer højst. Skemaplanlæggerne mener at et skemalægningsprogram ikke ville kunne hjælpe dem, da de programmer de har afprøvet ikke har kunne tage højde for flere parametre, og præferencer spredt ud over de forskellige teams og klasser. Samt at Sofiendahlskolen har lærer som underviser på flere klassetrin, hvilket gør at de forskellige skemaer er afhængige af hinanden, f.eks kan det være at hvis man rykker rundt på timer i 3. Klasses skema, skal man også rykke rundt på den i 5. Klasses skema da nogle af lærerne i 5. Klasses team også underviser i 3. Klasse osv.\footfullcite{interview}

Sofiendal skolen har prøvet at lægge skema med programmet tabulex. Tabulex var utrolig komplekst og avanceret at få sat op for hver eneste klasse da hver team har forskellige præferencer. Det var også svært for Tabulex at lægge skemaet da Tabulex ikke var fleksibelt nok til deres behov. Derfor har Sofiendal skolen valgt at gøre det på den gammeldags facon. Dog vil Sofiendal skolen stadig være åben for at bruge et computerprogram til at lægge skema, hvis programmet kunne hjælpe dem med at gøre processen kortere, uden at gøre det for avanceret at få sat op.\footfullcite{interview}

