For nemmere at lave ændringer i antallet af lærer, de timer de kan tage, og det antal timer de forskellige fag skal have, opstilles disse bindinger i en tekstfil. 
Elementerne i dokumentet står som følgende:
’Lærer forkortelse’ ’fag’ ’antal timer’ ’klasse’
Et eksempel på dette kunne være:
CA Dan 6 7a
JA Mat 4 7a
HA Eng 3 7a
RA Tys 3 7a
MA Fys 2 7a
MA His 2 7a
KA Sam 2 7a
UA Val 2 7a
RA Geo 2 7a
Dette bliver indlæst fra filen én gang i starten af programmet, hvorefter data’en fra dokumentet bliver gemt over i en struct.
Vi har gjort dette som følgende:
\begin{lstlisting}
void read_teacher_data(teacher teacher_data[][NUMBER_OF_FAG]){
 FILE *teacherinfo = fopen("teacherinfo.txt", "r");
  if(teacherinfo == NULL){
    perror("Error the file is empty");
    fclose(teacherinfo);
    exit(1);
  }

  teacher local_teacher_data;
  int i = 0, j = 0;
  for(j = 0; j < NUMBER_OF_CLASSES; j++){
    for(i = 0;i < NUMBER_OF_FAG; i++){
      fscanf(teacherinfo,
      " %s "
      "%s "
      " %d "
      "%s ",
      local_teacher_data.teacher_name, 
      local_teacher_data.lesson_name, 
      &local_teacher_data.number_of_lessons, 
      local_teacher_data.class_name);

      teacher_data[j][i] = local_teacher_data; 
  
    } 
  }  
  fclose(teacherinfo);
}
\end{lstlisting}
Først åbnes filen, hvorefter der bliver tjekket om filen er tom. Hvis dette er sandt, bliver der printet en fejl-besked ud, og programmet bliver lukket.
Derefter bliver der lavet Local_teacher_data af variablen ’teacher’, som er en struct. Denne bliver brugt i den efterfølgende løkke, som læser filen linje for linje, hvor den gemmer informationerne for hver linje, ind i local_teacher_data. Efter dette bliver det overført til ’teacher_data’ som er den endelige struct, med alt informationen omkring lærerne, deres dag, antal timer og deres klasse. 
Efter denne løkke er blevet kørt igennem det antal af gange der svarer til antallet af klasser bilver filen lukket igen, og alle informationer burde være kopieret over i ’teacher_data.’
