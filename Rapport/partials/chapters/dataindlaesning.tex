For at det er nemmere at lave ændringer i antallet af lærer, de timer de kan tage og det antal timer de forskellige fag skal have, er dette opstillet i en tekstfil. På figur~\ref{fig:datacollect} ses et grafisk diagram over funktionen.
\begin{figure}[!h]
\includegraphics[scale = 1]{partials/graphics/read_teachers_name.png}
\caption{Grafisk diagram over read\_teachers\_name}
\label{fig:datacollect}
\end{figure}
Elementerne i dokumentet står som følgende:

Lærerforkortelse fag antal timer klasse

Et eksempel på dette kunne være:

    CA Dan 6 7a

    JA Mat 4 7a

    HA Eng 3 7a

    RA Tys 3 7a

    MA Fys 2 7a

    MA His 2 7a

    KA Sam 2 7a

    UA Val 2 7a

    RA Geo 2 7a

Dette bliver indlæst fra filen én gang i starten af programmet, hvorefter data’en fra dokumentet bliver gemt over i et array af structs.

Dette er implementeret på følgende vis:

\begin{lstlisting}
void read_teachers_name(class_info **class_data){ 
  int i, j;
  class_info local_class_data;
  FILE *teacherinfo = fopen("teacherinfo.txt", "r");
  if(teacherinfo == NULL){
    perror("Error the file is empty");
    fclose(teacherinfo);
    exit(1);
  }

  for(j = 0; j < NUMBER_OF_CLASSES; j++){
    for(i = 0;i < NUMBER_OF_SUBJECTS; i++){
      fscanf(teacherinfo,
      " %s %s %d %s ",
      local_class_data.teacher_name, 
      local_class_data.lesson_name, 
      &local_class_data.number_of_lessons, 
      local_class_data.class_name);
      class_data[j][i] = local_class_data;
    } 
  }  
  fclose(teacherinfo);
}
\end{lstlisting}

Først åbnes filen, hvorefter det tjekkes om filen er tom. Hvis dette er sandt, bliver en fejl-besked printet ud, og programmet lukker igen.

Dernæst bliver der lavet local\_class\_data af typen ’class\_info’, som er et struct. Denne bliver brugt i den efterfølgende løkke, som læser filen linje for linje, hvor den gemmer informationerne for hver linje, ind i local\_class\_data. Efter dette bliver det overført til ’class\_info’ som er det endelige struct, med alt informationen omkring lærerne, deres dag, antal timer og deres klasse. Efter denne løkke bliver filen lukket igen.