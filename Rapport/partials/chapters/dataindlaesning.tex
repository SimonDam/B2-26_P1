For at det er nemmere lave ændringer i antallet af lærer, de timer de kan tage og det antal timer de forskellige fag skal have, er dette opstillet i en tekstfil. 
Elementerne i dokumentet står som følgende:

’Lærer forkortelse’ ’fag’ ’antal timer’ ’klasse’

Et eksempel på dette kunne være:

    CA Dan 6 7a
    JA Mat 4 7a
    HA Eng 3 7a
    RA Tys 3 7a
    MA Fys 2 7a
    MA His 2 7a
    KA Sam 2 7a
    UA Val 2 7a
    RA Geo 2 7a

Dette bliver indlæst fra filen én gang i starten af programmet, hvorefter data’en fra dokumentet bliver gemt over i en struct.
Dette er implementeret på følgende vis:

\begin{lstlisting}
void read_teacher_data(teacher teacher_data[][NUMBER_OF_SUBJECTS]){
 FILE *teacherinfo = fopen("teacherinfo.txt", "r");
  if(teacherinfo == NULL){
    perror("Error the file is empty");
    fclose(teacherinfo);
    exit(1);
  }

  teacher local_teacher_data;
  int i = 0, j = 0;
  for(j = 0; j < NUMBER_OF_CLASSES; j++){
    for(i = 0;i < NUMBER_OF_SUBJECTS; i++){
      fscanf(teacherinfo,
      " %s %s %d %s ",
      local_teacher_data.teacher_name, 
      local_teacher_data.lesson_name, 
      &local_teacher_data.number_of_lessons, 
      local_teacher_data.class_name);
      teacher_data[j][i] = local_teacher_data; 
  
    } 
  }  
  fclose(teacherinfo);
}
\end{lstlisting}

Først åbnes filen, hvorefter det tjekkes om filen er tom. Hvis dette er sandt, bliver der printet en fejl-besked ud, og programmet lukker igen.
Dernæst bliver der lavet Local\_teacher\_data af typen ’teacher’, som er et struct. Denne bliver brugt i den efterfølgende løkke, som læser filen linje for linje, hvor den gemmer informationerne for hver linje, ind i local\_teacher\_data. Efter dette bliver det overført til ’teacher\_data’ som er det endelige struct, med alt informationen omkring lærerne, deres dag, antal timer og deres klasse. 
Efter denne løkke bliver filen lukket igen, og alle informationer er kopieret over i ’teacher\_data’.