Et godt skema laver store krav til et skemalægningsprogram. Det er vigtigt at programmet er i stand til at møde disse krav, ellers er programmet essentielt ubrugeligt. Det grundlæggende princip i programmet er, at automatiserer skemalægningsprocessen ud fra forudbestemte parametre. Grundet den begrænsede tid, der er afsat til dette projekt, vil følgende afsnit afgrænse hvilke funktionaliteter, som vil blive løst i det endelige program.
Som tidligere nævnt vil programmets selektion være baseret på genetiske algoritmer. Brugen af genetiske algoritmer giver mulighed for at buge programmet gentagende gange med forskellige resultater. En bruger kan køre programmet indtil, der genereres et skema, de anser som tilfredsstillende.

Programmet skal tage højde for hvilke prædefineret krav, både lovmæssige og brugerspecifikke.

De lovmæssige krav er defineret i afsnittet om lovgivning angående. Der er ikke meget debat om at disse krav er essentielle grundsten, og derfor skal inkluderes for at lave et acceptabelt program. 

I interviewet med Søren Kusk blev konkrete problemer understeget. Forberedelses timer til lærerne, som ligger samlet, tunge fag skal ligge før middag, tværfaglig undervisning på tværs af klasser.\footfullcite{interview}

Programmet skal arbejde med at løse de tre problemer. Som nævnt i problemafgrænsningen er samlede forberedelsestimer et vigtigt krav at få opfyldt. Det er vigtigt for lærerne at have tilstrækkelig konsekutive forberedelses timer, så undervisningstimerne er af tilstrækkelig kvalitet for elevernes indlæring. Det er derfor medtaget i programmet, som et at de primære problemer, som skal løses. 

Et andet problem der er valgt til at løse, er problemet omkring tunge fag efter middag. Dette var et problem, som var blevet gjort til kende af Søren Kusk og som en præference, der ønskes at medtages i skemalægningen.\footfullcite{interview} Det er derfor også et krav, som vælges at tages højde for i forhold til programmet, både på grund af det var et problem værd at nævne af Søren Kusk, men ydermere et koncept genetiske algoritmer har potentiale til at løse ved hjælp af selektion og nedprioritering af skemaer med sådanne hændelser.

Sidste prioritet fra interviewet som vælges at programmet skal tage højde for, er kravet om tværfaglige undervisning på tværs af parallelklasserne. Der stilles krav om, at lærerne kan undervise parallelklasser samtidig. Dette er dog kun muligt, hvis to eller mere parallelklasser har det samme fag samtidig. Derfor skal programmet være i stand til vurdere placeringen af specifikke fag og lærere på tværs af parallelklasser. 

Disse krav er lavet ud fra interviewet, men yderligere krav blev opstillet. Kravene er som følgende:

Sofiendalskolen ønsker ikke at have mere end to blokke med samme fag i træk. \footfullcite{interview}

Programmet skal være i stand til at lave et mængde skemaer for flere klasser. Grundet omfanget af dette krav og den tidsmæssige begrænsning, der ligger på dette projekt, er valget truffet om at begrænse dette krav til 7. 8. og 9. klasse. 

Fritimer skal ligge i slutningen af dagen og bestemt ikke i midten af dagen. Dette er gjort for ikke, at forlænge skoledagen længere end nødvendigt, ydermere ikke have længere pauser, som kan lede i brud af koncentrationen hos eleverne.

Programmet skal bruge en fil, som simulerer brugerinput. Valget er taget som et alternativ til en brugergrænseflade.  
