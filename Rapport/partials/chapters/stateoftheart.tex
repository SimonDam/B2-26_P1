Skemalægningen har længe været et problem, som diverse skoler har haft svært ved at løse, heriblandt er Sofiendalskolen, som vi har interviewet.
Selvom der findes mange gode skemalægningsprogrammer er det ikke nødvendigvis ensbetydende med at alle skoler kan benytte disse programmer af den grund, at nogle skoler har svært ved at begrænse sig til så få parametre, som programmerne indeholder. Herudover nævner interviewpersonen, Søren Kusk, at: ”[…] selvom det tager højde for mange ting, så er der bare nogle ting som det ikke altid tager højde for.” Dette tydeliggør problematikken og pointen i, at skemalægningsprogrammerne ganske enkelt ikke indeholder nok parametre og er præcis nok, til at skoler med forhindringer kan gøre brug af programmerne. 