
Skemaernes fitness er beregnet ud fra værdierne vist på figur~\ref{fitnessvalues}.

\begin{figure}[!h]
\includegraphics[scale = 2]{partials/graphics/fitness.png}
\caption{Billede over de fitness værdier skemaerne bliver tildelt, i tilfælde af at de opfylder visse parametre}
\label{fitnessvalues}
\end{figure}
De forskellige fitnessparametre kan godt give negative eller positive værdier mere end en gang. Det vil sige at, hvis der er to steder på skemaet, hvor der er en lærer der er overbooked giver det -2000 i fitness to gange.
I denne test, undersøges det om fitnessen udregnes korrekt. De nedenstående skemaer er generet gennem softwareløsningen. De tre skemaer er alle skemaer for 7.C. De skal altså derfor alle have det samme antal timer og de samme fag. 
I det første skema mangler 7.C to dansk lektioner. Det kan ses på perfection grade, der er på 12, hvis skemaet havde opfyldt alle krav om timeantal, ville perfection graden have været 13. Udover at der mangler to dansk lektioner har dette skema også for mange af nogle lektioner. Dette kan observeres i "Must have" og "Have", hvor fagene står i følgende rækkefølge dansk, matematik, engelsk, tysk, fysik/kemi, historie, samfundsfag, valgfag, geografi, biologi, idræt, kristendom, praktiske fag og fri timer til sidst. I det første skema er der fem fag med for mange lektioner, navnligt engelsk, hvor der er to lektioner for meget, tysk hvor der er en lektion for meget, historie hvor der er en for meget, samfundsfag hvor der er en for meget og biologi hvor der er en for meget. Det er dog kun de fag, med to eller flere lektioner for meget, der har negativ indflydelse på fitnessniveaut.
Skemaets positive fitness er en kombination af lessons with parallel, som er det antal af fag der foregår på samme tid som en af parallelklasserne og lessons with both, hvor der er mulighed for samarbejde mellem alle tre parallelklasser på en gang. Dette skema har otte lektioner, hvor det er muligt at arbejde sammen med en af parallelklasserne. Dette skema har dog ingen lektioner på samme tid med begge parallelklasser. Derudover får dette skema mindre i fitness, ved at det har ’tunge’ fag som ligger over middag. I dette skema ligger otte tunge fag som ligger efter middag. Derudover er lærerne også overbooked fire gange i dette skema. Det vil sige at der er fire tilfælde, hvor en af klassens lærere har en eller flere lektioner på samme tid. Skemaets samlede fitness er 5561. 
\begin{figure}[!h]
\includegraphics[width=\textwidth]{partials/graphics/fitness1.png}
\caption{Billede af et generet skema}
\label{fitness1}
\end{figure}

Det andet skema har en perfection grade på 13. lle kravene for antallet af lektioner for de forskellige fag er altså opfyldt. Der er dog to fag hvor der er en lektion for meget, og et fag hvor der er to lektioner for meget. I skemaet er der tre fælles lektioner med to parallel klasser. Der er dog ingen lektioner, hvor der er mulighed for samarbejde med tre parallelklasser på samme tid. På skemaet er der syv lektioner med tunge fag over middag. Der er to steder, hvor en lærer er overbooked. Dette skemas samlede fitness er 10341.
\begin{figure}[!h]
\includegraphics[width=\textwidth]{partials/graphics/fitness2.png}
\caption{Billede af et generet skema}
\label{fitness2}
\end{figure}

Perfection graden er 13 i det tredje skema. Der er altså ikke for lidt lektioner af nogle fag. Der er dog tre fag, hvor der er for mange lektioner. På skemaet er der otte fælles lektioner med en parallelklasse, og en parallellektion for alle tre klasser. På det tredje skema er der fem gange, hvor et tungt fag ligger over middag. Dette er det mindste antal ud af de tre skemaer. Det forekommer ikke at lærerne i det tredje skema har to planlagte lektioner på samme tid. 
\begin{figure}[!h]
\includegraphics[width=\textwidth]{partials/graphics/fitness3.png}
\caption{Billede af et generet skema}
\label{fitness3}
\end{figure}

På tabellen med fitnessværdier, kan det ses at fitnessværdien bliver trukket meget ned af at have lærere der er ’overbooked’. Dette passer med, at det tredje skema er det bedste, da det ikke har nogle lærere der er ’overbooked’. Derudover er det også deT tredje skema der har færrest tunge lektioner over middag. Det sidste skema er også det eneste skema, der har en lektion med mulighed for samarbejde mellem alle parallelklasserne sammentidig. 
Ud fra tabellen med fitnessværdierne og undersøgelsen af de tre skemaer, kan det konkluderes at fitnessfunktionen fungerer som den skal. 
