Problemanalysen har givet et overblik over de forskellige aspekter af skemalægningen. Det er tydeligt, at variablerne som indgår i processen skal overvejes for at et godt produkt kan laves. De lovmæssige krav til skoleskemaet er en nødvendig del at konkretisere, da de sætter rammerne produktet skal arbejde indenfor. Interviewet gav indblik i, hvordan skemalæggerne vurderede processen, hvilke hensyn og præferencer der tages i brug, samt hvilke problemer der opstår. De øvrige afsnit vurderer, hvordan processen allerede er forsøgt behandlet, state of the art, samt hvilke interessenter der er medvirkende og påvirker processen.

Afsnittet viser, at problemer opstår i selve processen. I Sofiendalskolens situation er det klart, at det er mange variabler, der indgår. At være i stand til at vægte disse variabler mod hinanden giver dog en hvis kompleksitet. Det er derfor vigtigt at forstå denne kompleksitet i processen. Skulle programmet kunne løse alle problemerne, samt stadig overholde de lovmæssige krav, ville projektet hurtigt blive uoverskueligt. En afgrænsning af projektets fokus redegøres for i følgende afsnit for at indsnævre fokusset til en konkret og håndterbar problemstilling.
