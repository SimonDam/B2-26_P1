Ved selektion anvendes roulette metoden i programmet. Dette foregår ved, at alle individer først lægges sammen for at finde størrelsen på rouletten. Denne gemmes i variablen "sum", som er en integer. Herefter vælges et tilfældigt punkt på rouletten ved hjælp af rand() modolus summen. Punktet gemmes i field integer'en.

Herefter køres en for-løkke der starter fra bunden af rouletten og lægger hver individs fitness oveni summen af de forrige, hvorefter der tjekkes om punktet ligger under den nye værdi. Hvis dette er tilfældet, bliver det i'ende individ returneret fra selektionen.  

\begin{lstlisting}[langueage=c]

individual pick_individual(individual individuals[]){
  int i, sum = 0;
  int fitness_test = 0;

  for(i = 0; i < NUMBER_OF_INDIVIDUALS; i++){
    sum += individuals[i].fitness;
  }

  int field = rand() \% sum;

  for(i = 0; i < NUMBER_OF_INDIVIDUALS; i++){
    fitness_test += individuals[i].fitness;
    if(field < fitness_test){
      return individuals[i];
    }
  }
}

\end{lstlisting}

Med denne metodes sikres det, at alle individer har en chance for at blive valgt, men at de bedre skemaer, i forhold til fitnessniveauet, har større chance for at blive valgt.