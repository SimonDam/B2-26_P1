Skoleskemaet giver struktur til eleverne og lærernes hverdag. Hvis et godt skoleskema bliver planlagt i starten af skoleåret, vil både elever og lærer gavne af det. Et forståeligt og overskueligt skema vil tillade lærerne at fokuserer på planlægningen af deres undervisning og eleverne får derved får bedre undervisning. I tilfælde af at lærerne synes skemaet er uoverskueligt og ikke passer til deres kriterier, vil det betyde, at andre lærere bliver nødt til at gå på kompromis og bytte lektioner. Det er derfor en fordel, hvis et endeligt skema bliver lagt  i starten af skoleåret, så forvirring ikke bliver skabt i løbet af året.\footfullcite{interview}

Der er mange parametre, der skal tages højde for, når skoleskemaet planlægges. Det kan derfor være en vanskelig proces at overskue. Skemaplanlæggerne skal blandt andet tage hensyn til ledige lokaler og gruppearbejde på tværs af parallelle klasser. Der er allerede eksisterende programmer, som kan danne skoleskemaer, men skoler som Sofiendalskolen vælger alligevel at lægge deres skema i hånden hvert år. Sofiendalskolens proces er langvarig og kostelig, da den kræver, at alle 70 lærere og pædagoger på skolen er til stede, mens de diskuterer skemaets opbygning. Derfor undersøges, hvilke særparametre skolen stiller til deres skema, siden de mener, at de aktuelle software løsninger ikke er tilpas brugervenlige nok eller ikke opfylder deres krav.\footfullcite{interview}

Ud fra dette opstilles der et initierende spørgsmål som lyder:
\\\\
Hvilke parametre tages der højde for når skoleskemaet planlægges i folkeskolerne, og hvorledes kan et program hjælpe skolerne i deres skemalægningsproces?