Skoleskemaet giver struktur til eleverne og lærernes hverdag. Hvis et godt skoleskema bliver planlagt i starten af skoleåret, vil både elever og lærer gavne af det. Et forståeligt og overskueligt skema vil betyde at lærerne kun behøver at tænke på at planlægge deres undervisning, og at eleverne derved får bedre undervisning. I tilfælde af at lærerne synes skemaet er uoverskueligt og ikke passer til deres kriterier, vil det betyde at andre lærer bliver nød til at gå på kompromis og bytte lektioner. Det er derfor en fordel hvis der bliver lagt et endeligt skema i starten af skoleåret, så der ikke bliver skabt forvirring i løbet af året\cite{interview}. 

Der er dog mange parametre, der skal tages højde for når der planlægges skoleskema, det er derfor ikke en overskuelig proces. Skemaplanlæggeren skal tage hensyn til ledige lokaler, gruppearbejde på tværs af parallelle klasser såvel som elevernes udbytte. Der er allerede eksisterende programmer der kan danne skoleskemaer, men skoler som Sofiendalskolen vælger alligevel at lægge deres i hånden hvert år. Sofiendalskolens proces er langvarig og kostbar, da den kræver at alle 70 lærer på skolen er til stede mens de diskuterer skemaets opbygning. Der kan derfor undersøges, hvilke særparametre skolen stiller til deres skema, siden de mener at de aktuelle software løsninger ikke er tilpas brugervenlige nok eller ikke opfylder deres krav\cite{interview}. 

Ud fra dette der opstilles et initierende spørgsmål som lyder:

Hvilke parametre tages der højde for når der planlægges skoleskema i folkeskolerne, og hvorledes et program kan hjælpe skolerne i deres skemalægningsproces?
\newpage
