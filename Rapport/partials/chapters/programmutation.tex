Det følgende er en oversigt over mutationsfunktionen:
\begin{figure}[!h]
\includegraphics[width=\textwidth]{partials/graphics/mutation.png}
\caption{Grafisk diagram over funktionen for mutation}
\label{fig:diagrammutation}
\end{figure}

Mutation er til for at lave små ændringer i skemaet. Den skal kunne sørge for at der er mangfoldighed, således at skemaet ikke ender i et lokalt maksimum. 
Måden hvorpå dette foregår er ved at tage et individ, finde to helt tilfældige timer på skemaet og bytte disse ud. Der kommer her to variabler, der kan finpudses for at finde den bedste løsning. Der er en procentvis chance for at en mutation kan ske, og antal mutationer der maksimalt kan ske pr. individ. 
Her under kan koden til mutations funktionen, der er brugt i programmet, ses.
\begin{lstlisting}
void mutation(individual individuals[]){
  int i = 0, j = 0, ran1Day = 0, ran1Week = 0, ran2Day = 0, ran2Week = 0, chance = 0, mutations = 0, temp = 0;

  for(i = 0; i < NUMBER_OF_INDIVIDUALS; i++){
    chance = rand() % 100;
    mutations = rand() % MAX_MUTATIONS_PER_INDIVIDUAL;
    for (j = 0; j < mutations; j++){
      if (chance > CHANCE_OF_MUTATION){
        do {
          ran1Week = rand() % SCHOOL_DAYS_IN_WEEK;
          ran1Day = rand() % LESSONS_PER_DAY_MAX;
          ran2Week = rand() % SCHOOL_DAYS_IN_WEEK;
          ran2Day = rand() % LESSONS_PER_DAY_MAX;
        } while ((ran1Week == ran2Week) && (ran1Day == ran2Day));
        
        temp = individuals[i].individual_num[ran1Day][ran1Week];
        individuals[i].individual_num[ran1Day][ran1Week] =        individuals[i].individual_num[ran2Day][ran2Week];
        individuals[i].individual_num[ran2Day][ran2Week] = temp;
      }
    }
  }
}
\end{lstlisting}

Der bliver kørt gennem tre for-løkker. Den første tæller klassen op, så der først bliver lavet mutationer på 7.a, så 7.b osv. Dernæst bliver der kørt gennem endnu en for-lykke, som går igennem antallet af individer, hvorefter et tilfældig tal mellem 0 og genereres. Hvis det generede tal er mindre end den valgte chance for mutation, forekommer mutationen. Antallet af ændringer ved en mutation vælges tilfældigt mellem 0 og det højeste antal tilladte mutationer. 