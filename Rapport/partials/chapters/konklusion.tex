For at få indblik i bindingerne relateret til skemaplanlægningsprocessen er en skemaplanlægger fra Sofiendalskolen blevet interviewet. Interviewet gav indblik i de specifikke problemer, der opstår når de planlægger skemaer på Sofiendalskolen. Derudover gav interviewet også indblik i problemerne relateret til de eksisterende softwareløsninger. Det blev dog hurtigt tydeligt, at der er en stor mængden af bindinger skulle tages højde for i skemalægningsprocessen. Det blev derfor valgt at begrænse hvilke parametre, der skulle tages højde for i softwareløsningen. 

Et af de valgte bindinger, der skulle prioriteres højest, var lærernes forberedelsestid. På Sofiendalskolen synes de, det er vigtigt, at lærerne har to forberedelsestimer før hver lektioner, så det bliver prioriteret i softwareløsningen. Derudover bliver det prioriteret at parallel klasserne har mulighed for at arbejde sammen, og at der er så lidt tunge fag over middag som muligt. Programmet skal også overholde de lovmæssige krav om undervisningstimer for de individuelle fag, samt de forskellige fag hvert klassetrin skal have. 

I planlægningen til softwareløsningen blev det valgt at bruge genetisk algoritme grundet kompleksiteten af skemalægningsprocessen. Kompleksiteten var bundet i umuligheden i at finde et idéelt skoleskema, og genetiske algoritmer gav en løsning ved at tilnærme sig et idéelt skema. Derudover blev det vurderet hvilken selektions metode, der skulle bruges i den genetiske algoritme. Her blev roulette metoden valgt. Roulette metoden blev valgt, da det er den metode, hvor fitness har mest indflydelse på hvilke individer, der bliver valgt til reproduktion. 

Ud fra den skabte softwareløsning kan det konkluderes, at genetiske algoritmer er en effektiv metode til at generer skemaer. Det kan dog også konkluderes at den softwareløsningen der blev lavet, ikke virkede optimalt. Fitnessen når et plateau efter ca. 10 generation, og derefter er det tilfældigt om der bliver generet et godt eller et dårligt skema. Som diskuteret i vurderingsafsnittet kan denne fejl skyldes, at crossoveren skaber for meget tilfældighed i algoritmen. Det blev forsøgt at lave en softwareløsning hvor crossoveren var bedre, og at den genetiske algoritme gjorde at skemaerne blev bedre i takt med at der blev skabt flere generation. I denne løsning blev skemaer dog værre i forhold til den første løsning, men indivdernes fitness voksede i takt med at der blev skabt flere generationer. Der var dog ikke nok tid til at optimere den løsning så den genrerede skemaer, der opfyldte kravene. 

Derudover kræver softwareløsningen flere egenskaber, før det ville kunne blive brugt på en reel skole, disse mangler er blevet diskuteret i videreudviklings afsnittet. Den vigtigste af disse mangler er bruger input, for at simulere bruger input i programmet indlæses en datafil. Da de forskellige skolers bindinger og prioriteter er forskellige, skal der indgå bruger input for at programmet skal være fleksibelt nok til at kunne bruges på andre skoler end Sofiendalskolen.
