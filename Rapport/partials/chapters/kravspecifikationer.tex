Hvilke krav og bindinger skal vægtes i programmet:


•	Programmet skal generere skemaer for 9 klasser, 3 skemaer for henholdsvis 7, 8 og 9 klassetrin.


•	Programmet skal overholde minimumskravene for timetal i folkeskolen.


•	Programmet skal tage højde for at klassetrinene ikke har det samme antal fag eller samme type fag.


•	Det skal ikke være muligt for en lærer at have lektioner i to klasser på en gang.


•	Lærerne skal have mere end en forberedelsestime adgangen.


•	Der må ikke være tomme lektioner i midten eller starten af skemaet.


•	Samarbejde på tværs af parallelklasserne skal være muligt. 


•	Programmet skal læse lærernes initialer ind via en fil. Filen skal simulere en indstillingsmenu for forbrugeren.


•	Der må højest være 8 lektioner på en dag.


•	Lektionerne skal helst være ligeligt fordelt over alle ugedagene, således at der ikke er 3 dage med 8 lektioner og 2 dage med 4 lektioner.
