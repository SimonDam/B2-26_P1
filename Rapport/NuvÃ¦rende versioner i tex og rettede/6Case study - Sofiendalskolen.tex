\subsection{Case study - Sofiendalskolen}
For at få et aktuelt indblik i nogle af de problemstillinger der opstår når en skole lægger et skema har vi fremstillet et interview med en af skemalæggerne fra Sofiendalskolen.
Sofiendal skolen blev opført i 1911, der ligger i det sydlige (?). I dag er Sofiendal skolen en tresporet skole med en speciel ADHD klasse hvilket betyder at der på en årgang befinder sig 3 klasse a, b og c samt speciel klassen. Skolen rummer 70 lærere samt pædagoger som underviser 650 elever dagligt. 
Lørdag den 27. oktober interviewede vi Søren Kusk. Søren fungere som matematiklærer samt it-ansvarlig på sofiendalskolen og indgår i et team med (find ud af hvordan hans team ser ud) kollegaer, som ligger skema for x klasse. På sofiendalskolen er det ikke en bestemt person, som er ansvarlig for at lægge skema, men derimod mødes alle involverede pædagoger og lærer 2 onsdage i starten i året. Her afholdes 2 møder med en varighed på 3 timer, hvilket vil sige det tager ca 420 mandetimer for sofiendalskolen at lægge årets skema. Skemalægningen er en meget simpel men kompliceret proces på sofiendalskolen. Lærerne og pædagogerne lægger skema uden brug af computerprogrammer. De får tildelt hvilket fag og hvor mange timer de skal have og herefter får de farvede brikker, som repræsenter de fag de skal undervise i…… (Få det helt præcist hvordan det foregår. Altså hvordan de får tildelt klasser og hvordan de danner teamerne og hvordan de bestemmer hvor hvilket lærer skal være osv.)
Når der lægges skema på sofiendalskolen ønskes der, at lærerne har sammenhængende forberedelses timer sådan de ikke er spredt ud over hele ugen, da de mener de ikke får chancen for at forberede sig ordentligt hvis de kun har en forberedelses time af gangen. Derudover prioriterer de at eleverne ikke har tunge fag som f.eks matematik over middag, da eleverne tit er trætte på daværende tidspunkt og derfor får begrænset udbytte af undervisning og at der er mulighed for at lave tværfaglig undervisning på tværs af klassetrinene, hvilket vil sige at alle 3 parallelklasser a, b og c f skal have mulighed for at have dansk på samme tidspunkt hver mandag. Skemaplanlæggerne på Sofiendalskolen føler at det er besværligt at opfylde alle disse 3 kriterier på en gang, derfor går de kompromis med parametrene og vælger hvilke de prioriterer højst. Skemaplanlæggerne mener at et skemalægningsprogram ikke ville kunne hjælpe dem, da de programmer de har afprøvet ikke har kunne tage højde for flere parametre, og præferencer spredt ud over de forskellige teams og klasser. (går jeg ud fra men vil gerne have det bekræftet).
Grunden til Sofiendalskolen stadig ligger skemaer i hånden er at ingen programmer (find ud af hvilket og hvad der præcis var galt med de gamle programmer)
