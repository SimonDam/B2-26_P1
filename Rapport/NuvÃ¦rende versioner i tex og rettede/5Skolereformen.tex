\subsection{Skolereform}
I 2014 blev den nye skolereform introduceret i folkeskolerne. Reformen opstillede tre klare mål for folkeskolerne:[6]
-	Folkeskolen skal udfordre alle elever, så de bliver så dygtige, de kan.
-	Folkeskolen skal mindske betydningen af social baggrund for de faglige resultater.
-	Tilliden til og trivslen i folkeskolen skal styrkes blandt andet gennem respekt for professionel viden og praksis.
Målene bliver mødt ved længere skoledage, der giver de mindste elever en skoledag, som slutter omkring kl 14, 14:30 for fjerde til sjette klasse og 15 fra syvende til niende. Den ekstra skoletid skal bl.a. støtte eleverne i bedre faglig fordybelse ved særligt udfordrende fag ved brug af lektiehjælp. Udover dette, bliver flere timer introduceret i form dansk og matematik fra fjerde til niende klasse. Engelsk, andet fremmedsprog og tredje fremmedsprog skal introduceres i henholdsvis første, femte og syvende klasse.[6]
Skolereformen har yderligere fokus som hævning af de pædagogiske kompetencer hos lærerne.
